\documentclass[cyan]{elegantnote}

\author{刘通(LIU Tong) Ethan}
\email{1142595791@qq.com}
\zhtitle{数学笔记}
\entitle{Note de Mathématiques}
\version{1.00}
\myquote{生命之中最快乐的是拼搏,而非成功;生命之中最痛苦的是懒散,而非失败}
\logo{logo_2.pdf}
\cover{cover.pdf}

%green color
   \definecolor{main1}{RGB}{210,168,75}
   \definecolor{seco1}{RGB}{9,80,3}
   \definecolor{thid1}{RGB}{0,175,152}
%cyan color
   \definecolor{main2}{RGB}{239,126,30}
   \definecolor{seco2}{RGB}{0,175,152}
   \definecolor{thid2}{RGB}{236,74,53}
%cyan color
   \definecolor{main3}{RGB}{127,191,51}
   \definecolor{seco3}{RGB}{0,145,215}
   \definecolor{thid3}{RGB}{180,27,131}

\usepackage{makecell}
\usepackage{lipsum}
\usepackage{float}

\begin{document}
\maketitle
\tableofcontents
\chapter{规范格式示例}

\section{编译方式}
本模板基于book文类,所以book的选项对于本模板也是有效的。但是,只支持 \XeLaTeX{},编码为 UTF-8,
推荐使用 \TeX{}live编译。作者编写环境为Win8(64bit)+\TeX{}live 2013。

\section{文档缺陷}
\begin{enumerate}
\item 定理类的环境在我们这个模板中不能浮动,也不能跨页。
\item 某些环境不足,比如例子、假设、性质、结论等环境,在1.00版本中已经增加了这几个环境。
\item
\item
\item
\item
\item
\item
\item
\item
\item
\end{enumerate}

\section{插图示例}
\begin{figure}[H]
\centering
\includegraphics[width=0.8\textwidth]{happy.jpg}
\caption{Happiness,We have it!\label{figur:happy}}
\end{figure}

\section{字体颜色}
{\color{thid}这章还有这么大空间,忍不住插个图!}

\section{关于字体}
本文主要使用的字体如下
\begin{itemize}
\itemsep=3pt
\parskip=0pt
\item Adobe Garamond Pro
\item Minion Pro \& Myriad Pro
\item 方正字体
\item 华文中宋
\end{itemize}

\begin{note}
需要特别注意的是,如果笔记需要使用到抄录环境的,请重新修改字体,此版本并未为抄录环境设置合适字体,
本note环境的字体即为抄录环境使用到的字体。
\end{note}

\section{选项设置}
本文特殊选项设置共有2类,分为{\color{main}颜色}和{\color{main}数学字体}。

第一类为{\color{main}颜色}主题设置,内置3组颜色主题,分别为green(default),cyan,blue。默认为
green颜色主题。需要改变颜色的话请自行到elegantnote.cls文件内对颜色的RGB值进行修改。

第二类为{\color{main}数学字体}设置,有两个可选项,分别是computer modern 和 mtpro2字体,默认使
用cm字体,无需在类文件前加选项,调用mtpro2字体的方法为\verb|\documentclass[mtpro]{elegantnote}|

\begin{table}[H]
\centering
\begin{tabular}{ccccc}
\toprule	
	   & green & cyan & blue & 主要使用的环境\\ 
\midrule
main & \makecell{{\color{main1}\rule{1cm}{1cm}}}& \makecell{{\color{main2}\rule{1cm}{1cm}}}
&\makecell{ {\color{main3}\rule{1cm}{1cm}}}& newdef\\

seco &\makecell{ {\color{seco1}\rule{1cm}{1cm}}}& \makecell{{\color{seco2}\rule{1cm}{1cm}}}
&\makecell{ {\color{seco3}\rule{1cm}{1cm}}}&newthem \ newlemma \ newcorol\\

thid &\makecell{ {\color{thid1}\rule{1cm}{1cm}}}& \makecell{{\color{thid2}\rule{1cm}{1cm}}}
&\makecell{ {\color{thid3}\rule{1cm}{1cm}}}&newprop\\
\bottomrule
\end{tabular}
\caption{Elegant note 模板中的三套颜色主题\label{tab:color thm}}
\end{table}

\section{数学环境简介}
一般的数学环境:

考虑如下的随机动态规划问题
\begin{align*}
&\max(\min)\quad \mathbb{E}\int_{t_0}^{t_1}f(t,x,u)\,dt\\
&\quad\mbox{s.t.} \quad dx=g(t,x,u)dt+\sigma(t,x,u)dz\\
&\quad \hspace{2.em} k(0)=k_0\;\text{given}
\end{align*}

在我们这个模板中,定义了三大类环境
\begin{enumerate}
\item 定理类环境,包含标题和内容两部分。根据格式的不同分为3种
\begin{itemize}
\item {\color{main} newdef} 环境,含有一个可选项,编号以章节为单位;

\begin{newdef}[Wiener Process]
If $z$ is wiener process, then for any partition $t_0,t_1,t_2,\ldots$ of time interval, the 
random variables $z(t_1)-z(t_0),z(t_2)-z(t_1),\ldots$ are independently and normally distributed 
with zero means and variance $t_1-t_0,t_2-t_1,\ldots$
\end{newdef}

\item {\color{main}newthem、newlemma、newcorol} 环境,三者颜色一致,但是定理环境编号以章节为单位,
引理和推论为全文编号;

\begin{newthem}[勾股定理]
勾股定理的数学表达为
\[a^2+b^2=c^2\]
其中$a,b$为直角三角形的两条直角边长,$c$为直角三角形斜边长。
\end{newthem}

\begin{newthem}[勾股定理]
勾股定理的数学表达为
\[a^2+b^2=c^2\]
其中$a,b$为直角三角形的两条直角边长,$c$为直角三角形斜边长。
\end{newthem}

\begin{newlemma}
假设$V(\cdot,\cdot)$为值函数,则跟据最大值原理,有如下推论
\[
V(k,z)=\max\Big\{u\big(zf(k)-y\big)+\beta \mathbb{E}V(y,z^\prime)\Big\}
\]
\end{newlemma}

\begin{newlemma}
假设$V(\cdot,\cdot)$为值函数,则跟据最大值原理,有如下推论
\[
V(k,z)=\max\Big\{u\big(zf(k)-y\big)+\beta \mathbb{E}V(y,z^\prime)\Big\}
\]
\end{newlemma}

\begin{newlemma}
假设$V(\cdot,\cdot)$为值函数,则跟据最大值原理,有如下推论
\[
V(k,z)=\max\Big\{u\big(zf(k)-y\big)+\beta \mathbb{E}V(y,z^\prime)\Big\}
\]
\end{newlemma}

\begin{newcorol}
假设$V(\cdot,\cdot)$为值函数,则跟据最大值原理,有如下推论
\[
V(k,z)=\max\Big\{u\big(zf(k)-y\big)+\beta \mathbb{E}V(y,z^\prime)\Big\}
\]
\end{newcorol}

\item newprop 环境,含有可选项,编号以章节为单位。

\begin{newprop}[最优性原理]
如果$u^*$在$[s,T]$上为最优解,则$u^*$在$[s,T]$任意子区间都是最优解,假设区间为$[t_0,t_1]$的最优解为
$u^*$,则$u(t_0)=u^{*}(t_0)$,即初始条件必须还是在$u^*$上。
\end{newprop}

\end{itemize}
\item 证明类环境,有{\color{main}newproof、note} 环境,特点是,有引导符和引导词,并且证明环境有结束
标志。

\begin{newproof}
因为 $y^*=\alpha\beta z k^\alpha$,$V(k,z)=\alpha/1-\alpha\beta\ln k_0+1/1-\alpha\beta \ln 
z_0+\Delta$。
\begin{align*}
\text{右边}&=\Big\{u\big(zf(k)-y\big)+\beta \mathbb{E}V(y,z^\prime)\Big\}\\
&=\ln(zk^\alpha-\alpha\beta zk^\alpha)+\beta\mathbb{E}\Big[\frac{\alpha}{1-\alpha\beta}\ln 
y+\frac{1}{1-\alpha\beta}\ln z^\prime+\Delta\Big]\\
&=\ln(1-\alpha\beta)zk^\alpha+\beta\Big\{\mathbb{E}\big[\frac{\alpha}{1-\alpha\beta}\ln 
\alpha\beta z k^\alpha\big]+\frac{1}{1-\alpha\beta}\mathbb{E}[\ln z^\prime]+\Delta\Big\}
\end{align*}
利用$\mathbb{E}[\ln z^\prime]=0$,并将对数展开得
\begin{align*}
\text{右边}&=\ln (1-\alpha\beta)+\ln z+\alpha\ln k+\frac{\alpha\beta}{1-\alpha\beta}\big[\ln 
\alpha\beta+\ln z+\alpha\ln k\big]+\frac{\beta}{1-\alpha\beta}\mu+\beta \Delta\\
&=\frac{\alpha}{1-\alpha\beta}\ln k+\frac{1}{1-\alpha\beta}\ln z+\Delta
\end{align*}
所以$\text{左边}=\text{右边}$,证毕。
\end{newproof}

\begin{note}
需要特别注意的是,如果笔记需要使用到抄录环境的,请重新修改字体,此版本并未为抄录环境设置合适字体,
本note环境的字体即为抄录环境使用到的字体。
\end{note}

\item 示例环境,有{\color{main} example、assumption、conclusion} 环境,三者均以粗体的引导词为开头,
字体以灰色,和普通段落格式一致。
\begin{example}
	今天看到一则小幽默,是这样说的:{\color{main} 别人都关心你飞的有多高,只有我关心你的翅膀好不好吃!}
说多了都是泪啊!
\end{example}
\begin{assumption}
	今天看到一则小幽默,是这样说的:{\color{main} 别人都关心你飞的有多高,只有我关心你的翅膀好不好吃!}
说多了都是泪啊!
\end{assumption}
\begin{conclusion}
今天看到一则小幽默,是这样说的:{\color{main} 别人都关心你飞的有多高,只有我关心你的翅膀好不好吃!}
说多了都是泪啊!
\end{conclusion}
\begin{conclusion}
今天看到一则小幽默,是这样说的:{\color{main} 别人都关心你飞的有多高,只有我关心你的翅膀好不好吃!}
说多了都是泪啊!
\end{conclusion}

\end{enumerate}

\section{可编辑的字段}
在模板中,可以编辑的字段分别为作者\verb|\author|、\verb|\email|、\verb|\zhtitle|、\verb|\entitle|、
\verb|\version|。并且,可以根据自己的喜好把封面水印效果的\verb|cover.pdf|替换掉,以及封面中用到的
\verb|logo.pdf|。

\end{document}
