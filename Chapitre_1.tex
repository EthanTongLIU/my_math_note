\documentclass[math_note.tex]{subfiles}

\begin{document}
\chapter{Série Entière 幂级数}
\begin{note}
	Pour $n \in \mathbb{N}$, on pose $X^{n}:z \in \mathbb{C} \mapsto z^{n}$ .
\end{note}
\begin{newdef}
	Pour $(a_{n})_{n\in \mathbb{N}} \in \mathbb{C}^{\mathbb{N}}$, on 
	appelle série entière $((a_{n}X^{n}))$. On s'intéresse à la fonction 
	$$f : z \mapsto \sum_{n=0}^{+\infty} a_{n} z^{n}$$
\end{newdef}

\section{Rayon de convergence 收敛半径}
\begin{newdef}
	设$((a_{n}X^{n}))$是一个幂级数,$((a_{n}X^{n}))$的收敛半径(RCV)记为$R_{c}$,定义为:
	$$R_{c} = sup(\{|z|,z \in \mathbb{C} ~ et ~ ((a_{n}X^{n})) ~ converge\}) \in 
	\mathbb{R}_{+} \cup \{ + \infty\}$$ (bien défini car $((a_{n}0^{n}))$ converge).
\end{newdef}

\begin{note}
		定义$\forall r \in \mathbb{R}_{+}$
		$$D(0,r)=\{ z \in \mathbb{C} : |z| < r \}$$
		$$\bar{D}(0,r)=\{ z \in \mathbb{C} : |z| \leq r \}$$
		$$C(0,r)=\{ z \in \mathbb{C} : |z| = r \}$$
\end{note}

\begin{newlemma}[d'Abel]
设$((a_{n}X^{n}))$是一个幂级数,假设存在$R > 0$ 满足数列$(a_{n}R^{n})$是有界的,则$\forall 
r \in ]0,R[$, $((a_{n}X^{n}))$在$\bar{D}(0,r)$上CVN。
\end{newlemma}

\begin{newproof}
	
\end{newproof}

\begin{conclusion}
	设$z\in \mathbb{C}^{*}$满足$f(z)$有定义并且$((a_{n}z^{n}))$收敛,则$\forall z^{'} \in 
	\mathbb{C}$ 满足 $|z^{'}| < |z|$, $((a_{n}{z^{'}}^{n}))$ 绝对收敛(Converge absolument)。
\end{conclusion}

\begin{example}
	\begin{enumerate}
		\item 求$((X^{n}))$ 的收敛半径。$R_{c}=1$ 
		\item 求$((\dfrac{X^{n}}{n!}))$的收敛半径。$R_{c}=+\infty$ 因为 $\forall z \in 
		\mathbb{C}$, $((\dfrac{X^{n}}{n!}))$收敛于$e^{z}$。
	\end{enumerate}
\end{example}

\begin{newthem}
	设$((a_{n}X^{n}))$是一个幂级数,收敛半径为$R_{c}$,对于$z \in \mathbb{C}$,
	\begin{itemize}
		\item 若$|z|<R_{c}$,则$((a_{n}z^{n}))$绝对收敛。
		\item 若$|z|>R_{c}$,则$((a_{n}z^{n}))$无界。
	\end{itemize}
\end{newthem}

\begin{newproof}
	
\end{newproof}

\begin{newdef}
	我们称收敛圆的集合为$C(0,R_{c})$。
\end{newdef}

\begin{example}
	\begin{enumerate}
		\item $((X^{n}))$:对于$z \in \mathbb{U}(|z|=1)$,$((z^{n}))$发散。
		实际上,
		\begin{itemize}
			\item 若$z=1$,$((1^{n}))$发散。
			\item 若$z\neq1$,设$N \in \mathbb{N}$,$ \sum\limits^{N}_{n=0} z^{n} = 
			\dfrac{1-z^{N+1}}{1-z}$,$|z^{N+1}-z^{N}|=|z^{N}||z-1|$,其中$|z^{n}|=1,|z-1|>0$。
			所以$(z^{N+1})$发散,$((z^{n}))$发散。
		\end{itemize}
		\item 求$((\dfrac{X^{n}}{n^{2}}))$的收敛半径。
		设$z \in \mathbb{C}^{R}$,$n \in \mathbb{N}^{R}$
		\begin{itemize}
			\item 若$|z|\leq1$,则$\forall n \in \mathbb{N}^{*}$,$|\dfrac{z^{n}}{n^{2}}|
			 \leq \dfrac{1}{n^{2}}$。
			\item 若$|z|>1$,$(\dfrac{z^{n}}{n^{2}})$无界。
		\end{itemize}
		\begin{conclusion}
			$Rc=1$,$\forall z \in C(0,1)$,$((\dfrac{z^{n}}{n^{2}}))$收敛。
		\end{conclusion}
	\end{enumerate}
\end{example}

\subsection{求收敛半径的方法}
\begin{itemize}
	\item 如果可以找到 $z_{0} \in \mathbb{C}$ 满足数列 $(a_{n}z_{0}^{n})$是有界的,则 $R_{c} 
	\geq |z_{0}|$。
	\item 如果级数$((a_{0}z_{0}^{n}))$发散或者数列$(a_{n}z_{0}^{n})$是无界的,则$R_{c} \leq
	 |z_{0}|$。
\end{itemize}

\newpage
练习题:求以下幂级数的收敛半径。
\begin{enumerate}
	\item $a_{n}=\dfrac{n^{n}}{n!}$(用两种方法)
	\begin{itemize}
		\item 方法1:直接使用\verb|d'Alembert|判别法。
		$$|\dfrac{a_{n+1}}{a_{n}}| = (1+\dfrac{1}{n})^{n} \underset{n \rightarrow +\infty}
		{\rightarrow} e$$
		则$((a_{n}X^{n}))$的收敛半径$R_{c}=\dfrac{1}{e}$。
		\item 方法2:根据\verb|STIRLING|定理,$n!\underset{n \rightarrow +\infty}{\sim} 
		\sqrt{2\pi n}(\dfrac{n}{e})^{n}$。则
		$$a_{n} \underset{n \rightarrow +\infty}{\sim} \dfrac{e^{n}}{\sqrt{2 \pi n}} $$
		则根据命题,找级数$ ((\dfrac{e^{n}}{\sqrt{2 \pi n}} X^{n})) $的收敛半径。
		$$ |\dfrac{\dfrac{e^{n+1}}{\sqrt{2 \pi (n+1)}}}{\dfrac{e^{n}}{\sqrt{2 \pi n}}}| 
		= |\dfrac{e\sqrt{n}}{\sqrt{n+1}}| \underset{ n \rightarrow +\infty}{\rightarrow}
		e$$
		则收敛半径是$\dfrac{1}{e}$。
	\end{itemize}
	
	\item $a_{n}=1+\dfrac{1}{n}$

	$1+\dfrac{1}{n} \underset{n \rightarrow +\infty}{\sim} 1$,根据命题,$(((1+\dfrac{1}{n}) 
	X^{n}))$和$((X^{n}))$相同,$R_{c}=1$。

	\item $a_{n}= \prod\limits_{k=1}^{n}(\dfrac{3k+1}{k+\dfrac{1}{2}})$

	$|\dfrac{a_{n+1}}{a_{n}}| \underset{n\rightarrow+\infty}{\rightarrow} 3$,$R_{c}=\dfrac{1}{3}$。

	\item $a_{n}=2+cos n$

	若$|z|=1$:有级数$((2+cosn))$发散,所以$R_{c}\leq 1$;并且有数列$(2+cosn)$是有界的,则$R_{c}
	 \geq 1 $。故$R_{c} = 1$。

	\item $a_{n}=C_{2n}^{n}$
    
    $|\dfrac{a_{n+1}}{a_{n}}|=|\dfrac{4n+2}{n+1}| \underset{n \rightarrow +\infty}{\rightarrow}
     4$,故$R_{c}=\dfrac{1}{4}$。

		
\end{enumerate}

以上是对收敛半径的一些总结,不全面,日后补充。

接下来是对有关幂级数的运算的一些总结。

\section{Opération sur les séries entières 幂级数的运算}

这里重点是对幂级数的和函数进行求导和积分运算。在之前还有幂级数的和与积的概念。

\begin{newprop}[幂级数的和与积]
	设$((a_{n}X^{n}))$和$((b_{n}X^{n}))$是两个幂级数,定义数列:
	$$(s_{n})=(a_{n}+b_{n})_{n \in \mathbb{N}} \quad (p_{n})=(\sum\limits_{k=0}^{n}
	 a_{k}b_{n-k})_{n \in \mathbb{N}}$$

	那么有两个幂级数的和为:
	$$\sum\limits_{n=0}^{+\infty} s_{n}z^{n} = \sum\limits_{n=0}^{+\infty} a_{n}z^{n}
	 + \sum\limits_{n=0}^{+\infty} b_{n}z^{n} = \sum\limits_{n=0}^{+\infty} (a_{n}+b_{n})
	 z^{n}$$

	而两个幂级数的乘积定义为:
	$$\sum\limits_{n=0}^{+\infty}p_{n}z^{n} =  (\sum\limits_{n=0}^{+\infty}a_{n}z^{n})
	(\sum\limits_{n=0}^{+\infty}b_{n}z^{n}) =\sum\limits_{n=0}^{+\infty}(\sum\limits
	_{k=0}^{n} a_{k}b_{n-k})z^{n}$$

\end{newprop}

接下来讨论对于幂级数的和函数进行求导和积分的运算。

在幂级数的收敛区间$]-R_{c},R_{c}[$内,幂级数的和函数定义为:
$$f:x \in ]-R_{c},R_{c}[ \mapsto \sum\limits_{n=0}^{+\infty}a_{n}x^{n}$$
\begin{multicols}{2}
对其进行逐项求导得到:
$$ \sum\limits_{n=0}^{+\infty}(n+1)a_{n+1}x^{n} $$
也就是级数 $ (((n+1)a_{n+1}x^{n}))_{n \in \mathbb{N}} $。
\\
逐项求积分得到:
$$ \sum\limits_{n=0}^{+\infty}\dfrac{a_{n}}{n+1}x^{n+1} $$
也就是级数 $ ((\dfrac{a_{n}}{n+1}x^{n+1}))_{n \in \mathbb{N}} $。
\end{multicols}

有结论级数$ (((n+1)a_{n+1}x^{n}))_{n \in \mathbb{N}} $ 和 $ ((\dfrac{a_{n}}
{n+1}x^{n+1}))_{n \in \mathbb{N}} $ 与级数  $ ((a_{n}x^{n})) $ 的收敛半径是相同的。

而且还有和函数 $f$ 在幂级数的收敛区间 $]-R_{c},R_{c}[$ 可导,$\forall x \in 
]-R_{c},R_{c}[$,
$$ f'(x) = \sum\limits_{n=0}^{+\infty}(n+1)a_{n+1}x^{n} $$ 
$f$ 在 $0$ 与 $x$ 这个区间上可积,且有:
$$ \int_{0}^{x}f(t)dt =  \sum\limits_{n=0}^{+\infty}\dfrac{a_{n}}{n+1}x^{n+1} $$

\begin{example}
	几何级数$ ((x^{n})) $在收敛域 $(-1,1)$ 内有 $\sum\limits_{n=0}^{+\infty}x^{n}
	 = \dfrac{1}{1-x} $。

	对级数$ ((x^{n})) $在收敛域 $(-1,1)$ 内逐项求导得到:
	$$ \sum\limits_{n=0}^{+\infty}(n+1)x^{n} = \dfrac{1}{(1-x)^{2}} $$
	$$ \sum\limits_{n=1}^{+\infty}n(n+1)x^{n-1} = \dfrac{2!}{(1-x)^{3}} $$
	对级数$ ((x^{n})) $在$[0,x[(x<1)$上逐项求积分可得:
	$$ \int_{0}^{x} \dfrac{1}{1-t} dt = \sum\limits_{n=0}^{+\infty}\dfrac{x^{n+1}}
	{n+1} = ln \dfrac{1}{1-x} $$

\end{example}

\end{document}















